\documentclass[fleqn, a4paper, 11pt, oneside]{amsart}
%\usepackage[top = 2cm, bottom = 1cm, left = 1cm, right = 1cm]{geometry}
\usepackage{exsheets, tasks}
\usepackage{amsmath, amssymb, amsthm} %standard AMS packages
\usepackage{marginnote} %marginnotes
\usepackage{gensymb} %miscellaneous symbols
\usepackage{commath} %differential symbols
\usepackage{xcolor} %colours
\usepackage{cancel} %cancelling terms
\usepackage[free-standing-units, space-before-unit]{siunitx} %formatting units
\usepackage{tikz, pgfplots} %diagrams
\usetikzlibrary{calc, hobby, patterns, intersections, decorations.markings}
\usepackage{graphicx} %inserting graphics
\usepackage{hyperref} %hyperlinks
\usepackage{datetime} %date and time
\usepackage{ulem} %underline for \emph{}
\usepackage{xfrac} %inline fractions
\usepackage{enumerate,enumitem} %numbered lists
\usepackage{float} %inserting floats
\usepackage{circuitikz}[american voltages, american currents] %circuit diagrams

\newcommand\numberthis{\addtocounter{equation}{1}\tag{\theequation}} %adds numbers to specific equations in non-numbered list of equations

\newcommand{\AxisRotator}[1][rotate=0]{
	\tikz [x=0.25cm,y=0.60cm,line width=.2ex,-stealth,#1] \draw (0,0) arc (-150:150:1 and 1);%
} %rotation symbols on axes

\theoremstyle{definition}
\newtheorem{example}{Example}
\newtheorem{definition}{Definition}

\theoremstyle{theorem}
\newtheorem{theorem}{Theorem}

\newcommand{\curl}{\mathrm{curl\,}}

\makeatletter
\@addtoreset{section}{part} %resets section numbers in new part
\makeatother

\renewcommand{\thesubsection}{(\arabic{subsection})}
\renewcommand{\thesection}{(\arabic{section})}

%section headings on left
\makeatletter
\def\specialsection{\@startsection{section}{1}%
	\z@{\linespacing\@plus\linespacing}{.5\linespacing}%
	%  {\normalfont\centering}}% DELETED
	{\normalfont}}% NEW
\def\section{\@startsection{section}{1}%
	\z@{.7\linespacing\@plus\linespacing}{.5\linespacing}%
	%  {\normalfont\scshape\centering}}% DELETED
	{\normalfont\scshape}}% NEW
\makeatother

%forces newline after subsection
\makeatletter
\def\subsection{\@startsection{subsection}{3}%
	\z@{.5\linespacing\@plus.7\linespacing}{.1\linespacing}%
	{\normalfont\itshape}}
\makeatother

\settasks{counter-format = tsk[1].}

\SetupExSheets{solution/print = true}

%opening
\title{Harmonic Analysis : Assignment 6}
\author
{
	Aakash Jog\\
	ID : 989323563
}
\date{\formatdate{15}{12}{2015}}

\begin{document}

\tikzset{->-/.style={decoration={
  markings,
  mark=at position #1 with {\arrow{>}}},postaction={decorate}}}

\maketitle
%\setlength{\mathindent}{0pt}

\begin{question}
	Calculate as exactly as you can how many times the series is continuously differentiable.
	\begin{enumerate}
		\item $f(x) = \sum\limits_{n = 1}^{\infty} \frac{\sin(n x)}{n^4}$
		\item $f(x) = \sum\limits_{n = 1}^{\infty} \frac{\sin(n x)}{n^{4.01}}$
		\item $f(x) = \sum\limits_{n = 1}^{\infty} \frac{1}{n^4} e^{i n x}$
		\item $f(x) = \sum\limits_{n = 1}^{\infty} \frac{1}{n^{5.01}} e^{i n x}$
		\item $f(x) = \sum\limits_{n = 0}^{\infty} \frac{1}{2^n} e^{i 2^n x}$
		\item $f(x) = \sum\limits_{n = 2}^{\infty} \frac{e^{i n x}}{n \log^2 n}$
		\item $f(x) = \sum\limits_{n = 2}^{\infty} \frac{e^{i n x}}{n \log n}$
	\end{enumerate}
\end{question}

\begin{solution}
	\begin{enumerate}[leftmargin=*]
		\item
			\begin{align*}
				\lim\limits_{n \to \infty} n \frac{\sin(n x)}{n^4}   & = 0    \\
				\lim\limits_{n \to \infty} n^2 \frac{\sin(n x)}{n^4} & = 0    \\
				\lim\limits_{n \to \infty} n^3 \frac{\sin(n x)}{n^4} & = 0    \\
				\lim\limits_{n \to \infty} n^4 \frac{\sin(n x)}{n^4} & \neq 0 \\
			\end{align*}
			Therefore, the function is continuously differentiable at most 3 times.
		\item
			\begin{align*}
				\lim\limits_{n \to \infty} n \frac{\sin(n x)}{n^{4.01}}   & = 0 \\
				\lim\limits_{n \to \infty} n^2 \frac{\sin(n x)}{n^{4.01}} & = 0 \\
				\lim\limits_{n \to \infty} n^3 \frac{\sin(n x)}{n^{4.01}} & = 0 \\
				\lim\limits_{n \to \infty} n^4 \frac{\sin(n x)}{n^{4.01}} & = 0 \\
				\lim\limits_{n \to \infty} n^5 \frac{\sin(n x)}{n^{4.01}} & \neq 0
			\end{align*}
			Therefore, the function is continuously differentiable at most 4 times.
		\item
			\begin{align*}
				\lim\limits_{n \to \infty} n \frac{e^{i n x}}{n^4}   & = 0    \\
				\lim\limits_{n \to \infty} n^2 \frac{e^{i n x}}{n^4} & = 0    \\
				\lim\limits_{n \to \infty} n^3 \frac{e^{i n x}}{n^4} & = 0    \\
				\lim\limits_{n \to \infty} n^4 \frac{e^{i n x}}{n^4} & \neq 0 \\
			\end{align*}
			Therefore, the function is continuously differentiable at most 3 times.
			\begin{align*}
				\frac{1}{n^4} & \le \frac{1}{n^{2 + 1 + \varepsilon}} \\
			\end{align*}
			Therefore, the function is differentiable 2 times.
		\item
			\begin{align*}
				\lim\limits_{n \to \infty} n \frac{e^{i n x}}{n^{5.01}}   & = 0 \\
				\lim\limits_{n \to \infty} n^2 \frac{e^{i n x}}{n^{5.01}} & = 0 \\
				\lim\limits_{n \to \infty} n^3 \frac{e^{i n x}}{n^{5.01}} & = 0 \\
				\lim\limits_{n \to \infty} n^4 \frac{e^{i n x}}{n^{5.01}} & = 0 \\
				\lim\limits_{n \to \infty} n^5 \frac{e^{i n x}}{n^{5.01}} & = 0 \\
				\lim\limits_{n \to \infty} n^6 \frac{e^{i n x}}{n^{5.01}} & \neq 0
			\end{align*}
			Therefore, the function is continuously differentiable at most 5 times.
			\begin{align*}
				\frac{1}{n^{5.01}} & \le \frac{1}{n^{4 + 1 + \varepsilon}} \\
			\end{align*}
			Therefore, the function is differentiable 4 times.
		\item
			Let
			\begin{align*}
				k & = 2^n
			\end{align*}
			Therefore,
			\begin{align*}
				f(x) & = \sum\limits_{k = 1,2,4,\dots}^{\infty} \frac{1}{k} e^{i k x}
			\end{align*}
			Therefore, $f(x)$ is differentiable 0 times.\\
		\item
			\begin{align*}
				\lim\limits_{n \to \infty} n \frac{e^{i n x}}{n \log^2 n}   & = 0 \\
				\lim\limits_{n \to \infty} n^2 \frac{e^{i n x}}{n \log^2 n} & \neq 0
			\end{align*}
			Therefore, the function is continuously differentiable at most 2 times.
		\item
			\begin{align*}
				\lim\limits_{n \to \infty} n \frac{e^{i n x}}{n \log n}   & = 0 \\
				\lim\limits_{n \to \infty} n^2 \frac{e^{i n x}}{n \log n} & \neq 0
			\end{align*}
			Therefore, the function is continuously differentiable at most 2 times.
	\end{enumerate}
\end{solution}

\begin{question}
	Let
	\begin{align*}
		f(x) & = 3 x^5 - 10 \pi^2 x^3 + 7 \pi^4 x
	\end{align*}
	Prove
	\begin{align*}
		\lim\limits_{n \to \infty} n^3 c_n & = 0
	\end{align*}
	where $c_n$ are the Fourier coefficients of $f(x)$.
\end{question}

\begin{solution}
	The function is 5 times differentiable.\\
	Therefore,
	\begin{align*}
		\lim\limits_{n \to \infty} \left| n^5 c_n \right| & = 0 \\
		\lim\limits_{n \to \infty} \left| n^3 c_n \right| & = 0
	\end{align*}
\end{solution}

\begin{question}
	For the function
	\begin{align*}
		f(x) & = \sum\limits_{n = 1}^{\infty} \left( \frac{\sin n x}{3 n^4 + 5 n^6} + \frac{\cos n x}{n^9 + 3 n^7} \right)
	\end{align*}
	Find the maximal $k$, such that $f^{(k)}(x)$ is continuous.
\end{question}

\begin{solution}
	\begin{align*}
		\lim\limits_{n \to \infty} n f(x)   & = 0 \\
		\lim\limits_{n \to \infty} n^2 f(x) & = 0 \\
		\lim\limits_{n \to \infty} n^3 f(x) & = 0 \\
		\lim\limits_{n \to \infty} n^4 f(x) & = 0 \\
		\lim\limits_{n \to \infty} n^5 f(x) & \neq 0
	\end{align*}
	Therefore, $f$ is differentiable 4 times.\\
	Therefore, $f^{(4)}(x)$ exists.\\
	Therefore, $f^{(3)}(x)$ must be differentiable, and hence continuous.\\
	Therefore, the maximal $k$ is $3$.
\end{solution}

\end{document}
