\documentclass[fleqn, a4paper, 11pt, oneside]{amsart}
%\usepackage[top = 2cm, bottom = 1cm, left = 1cm, right = 1cm]{geometry}
\usepackage{exsheets, tasks}
\usepackage{amsmath, amssymb, amsthm} %standard AMS packages
\usepackage{marginnote} %marginnotes
\usepackage{gensymb} %miscellaneous symbols
\usepackage{commath} %differential symbols
\usepackage{xcolor} %colours
\usepackage{cancel} %cancelling terms
\usepackage[free-standing-units, space-before-unit]{siunitx} %formatting units
\usepackage{tikz, pgfplots} %diagrams
\usetikzlibrary{calc, hobby, patterns, intersections, decorations.markings}
\usepackage{graphicx} %inserting graphics
\usepackage{hyperref} %hyperlinks
\usepackage{datetime} %date and time
\usepackage{ulem} %underline for \emph{}
\usepackage{xfrac} %inline fractions
\usepackage{enumerate,enumitem} %numbered lists
\usepackage{float} %inserting floats
\usepackage{circuitikz}[american voltages, american currents] %circuit diagrams

\newcommand\numberthis{\addtocounter{equation}{1}\tag{\theequation}} %adds numbers to specific equations in non-numbered list of equations

\newcommand{\AxisRotator}[1][rotate=0]{
	\tikz [x=0.25cm,y=0.60cm,line width=.2ex,-stealth,#1] \draw (0,0) arc (-150:150:1 and 1);%
} %rotation symbols on axes

\theoremstyle{definition}
\newtheorem{example}{Example}
\newtheorem{definition}{Definition}

\theoremstyle{theorem}
\newtheorem{theorem}{Theorem}

\newcommand{\curl}{\mathrm{curl\,}}

\makeatletter
\@addtoreset{section}{part} %resets section numbers in new part
\makeatother

\renewcommand{\thesubsection}{(\arabic{subsection})}
\renewcommand{\thesection}{(\arabic{section})}

%section headings on left
\makeatletter
\def\specialsection{\@startsection{section}{1}%
	\z@{\linespacing\@plus\linespacing}{.5\linespacing}%
	%  {\normalfont\centering}}% DELETED
	{\normalfont}}% NEW
\def\section{\@startsection{section}{1}%
	\z@{.7\linespacing\@plus\linespacing}{.5\linespacing}%
	%  {\normalfont\scshape\centering}}% DELETED
	{\normalfont\scshape}}% NEW
\makeatother

%forces newline after subsection
\makeatletter
\def\subsection{\@startsection{subsection}{3}%
	\z@{.5\linespacing\@plus.7\linespacing}{.1\linespacing}%
	{\normalfont\itshape}}
\makeatother

\settasks{counter-format = tsk[1].}

\SetupExSheets{solution/print = true}

%opening
\title{Harmonic Analysis : Assignment 4}
\author
{
	Aakash Jog\\
	ID : 989323563
}
\date{\formatdate{24}{11}{2015}}

\begin{document}

\tikzset{->-/.style={decoration={
  markings,
  mark=at position #1 with {\arrow{>}}},postaction={decorate}}}

\maketitle
%\setlength{\mathindent}{0pt}

\begin{question}
	Let
	\begin{align*}
		g(x) &=
			\begin{cases}
				\cos x & ;\quad -\pi \le x \le 0 \\
				\sin x & ;\quad 0 < x \le \pi    \\
			\end{cases}
	\end{align*}
	\begin{enumerate}
		\item
			Calculate Fourier series of $g$.
		\item
			Let us define
			\begin{align*}
				h(x) & = \int\limits_{-\pi}^{x} g(t) \dif t + a \sin \frac{x}{2}
			\end{align*}
			For which values of $a$, will the Fourier series of $h$ uniformly converge?
	\end{enumerate}
\end{question}

\begin{solution}
	\begin{enumerate}[leftmargin=*]
		\item
			\begin{align*}
				g(x) &=
					\begin{cases}
						\cos x & ;\quad -\pi \le x \le 0 \\
						\sin x & ;\quad 0 < x \le \pi    \\
					\end{cases}
			\end{align*}
			Therefore,
			\begin{align*}
				a_0 & = \frac{1}{\pi} \int\limits_{-\pi}^{\pi} g(x) \dif x                                                     \\
                                    & = \frac{1}{\pi} \int\limits_{-\pi}^{0} \cos x \dif x + \frac{1}{\pi} \int\limits_{0}^{\pi} \sin x \dif x \\
                                    & = \frac{2}{\pi}
			\end{align*}
			\begin{align*}
				a_n &= \frac{1}{\pi} \int\limits_{-\pi}^{\pi} g(x) \cos(n x) \dif x\\
				&= \frac{1}{\pi} \int\limits_{-\pi}^{0} \cos x \cos(n x) \dif x + \frac{1}{\pi} \int\limits_{0}^{\pi} \sin x \cos(n x) \dif x\\
				&=
					\begin{cases}
						\frac{1}{2}                                   & ;\quad n = 1    \\
						\frac{1 + (-1)^n}{\pi \left( n^2 - 1 \right)} & ;\quad n \neq 1 \\
					\end{cases}
			\end{align*}
			\begin{align*}
				b_n &= \frac{1}{\pi} \int\limits_{-\pi}^{\pi} g(x) \sin(n x) \dif x\\
				&= \frac{1}{\pi} \int\limits_{-\pi}^{0} \cos x \sin(n x) \dif x + \frac{1}{\pi} \int\limits_{0}^{\pi} \sin x \sin(n x) \dif x\\
				&=
					\begin{cases}
						\frac{1}{2}                                  & ;\quad n = 1    \\
						-\frac{n \left( 1 + (-1)^n \right)}{n^2 - 1} & ;\quad n \neq 1 \\
					\end{cases}
			\end{align*}
			Therefore,
			\begin{align*}
				a_{2 n} & = \frac{2}{\pi \left( 1 - 4 n^2 \right)}   \\
				b_{2 n} & = \frac{4 n}{\pi \left( 1 - 4 n^2 \right)} \\
			\end{align*}
			Therefore,
			\begin{align*}
				g(x) & \approx \frac{1}{\pi} + \frac{\sin x + \cos x}{2} + \sum\limits_{n = 1}^{\infty} \left( \frac{2 \cos(2 n x)}{\pi \left( 1 - 4 n^2 \right)} + \frac{4 n \sin(2 n x)}{\pi \left( 1 - 4 n^2 \right)} \right)
			\end{align*}
		\item
			\begin{align*}
				h(x) &= \int\limits_{-\pi}^{x} g(t) \dif t + a \sin \frac{x}{2}
			\end{align*}
			Therefore,
			\begin{align*}
				h(x) &=
					\begin{cases}
						\int\limits_{-\pi}^{x} \cos t \dif t + a \sin \frac{x}{2}                                     & ;\quad -\pi \le x \le 0 \\
						\int\limits_{-\pi}^{0} \cos t \dif t + \int\limits_{0}^{x} \sin t \dif t + a \sin \frac{x}{2} & ;\quad 0 < x \le \pi    \\
					\end{cases}\\
				&=
					\begin{cases}
						\sin x + a \sin \frac{x}{2}     & ;\quad -\pi \le x \le 0 \\
						1 - \cos x + a \sin \frac{x}{2} & ;\quad 0 < x \le \pi    \\
					\end{cases}
			\end{align*}
			Therefore, for the Fourier series to converge uniformly, $h(-\pi) = h(\pi)$.\\
			Therefore,
			\begin{align*}
				\sin(-\pi) + a \sin \frac{-\pi}{2} & = 1 - \cos \pi + a \sin \frac{\pi}{2}
			\end{align*}
			Therefore,
			\begin{align*}
				a & = 0
			\end{align*}
	\end{enumerate}
\end{solution}

\begin{question}
	Let $\hat{f}(n)$ represent coefficients.
	Let $\sum\limits_{n = -\infty}^{\infty} \hat{f}(n) e^{i n x}$ be the Fourier series of $f(x) = |x|$.
	Prove that the series $\sum\limits_{n = -\infty}^{\infty} n \hat{f}(n) e^{i n x}$ converges for all $x$ on $[-\pi,\pi]$.
\end{question}

\begin{solution}
	$f(x)$ is continuous, $f'(x)$ is piecewise continuous, and $f(-\pi) = f(\pi)$.\\
	Therefore, differentiating term by term,
	\begin{align*}
		f'(x) & \approx i \sum\limits_{n = -\infty}^{\infty} n \hat{f}(n) e^{i n x}
	\end{align*}
	Therefore, the series $\sum\limits_{n = -\infty}^{\infty} n \hat{f}(n) e^{i n x}$ converges for all $x \in [-\pi,\pi]$.
\end{solution}

\begin{question}
	Let
	\begin{align*}
		f(x) &=
			\begin{cases}
				\sin 2 x & ;\quad -\frac{\pi}{2} < x < \frac{\pi}{2} \\
				0        & ;\quad \text{otherwise}                   \\
			\end{cases}
	\end{align*}
	\begin{enumerate}
		\item Find the Fourier series of $f$ and $f'$.
		\item Where does the Fourier series of $f'$ converge at $x = \pm \frac{\pi}{2}$?
	\end{enumerate}
\end{question}

\begin{solution}
	\begin{enumerate}[leftmargin=*]
		\item
			\begin{align*}
				f(x) &=
					\begin{cases}
						\sin 2 x & ;\quad -\frac{\pi}{2} < x < \frac{\pi}{2} \\
						0        & ;\quad \text{otherwise}                   \\
					\end{cases}
			\end{align*}
			Therefore, as $f(x)$ is odd, its Fourier series is
			\begin{align*}
				f(x) & \approx \sum\limits_{n = 1}^{\infty} b_n \sin(n x)
			\end{align*}
			Therefore,
			\begin{align*}
				b_n & = \frac{1}{\pi} \int\limits_{-\frac{\pi}{2}}^{\frac{\pi}{2}}
			\end{align*}
			Therefore,
			\begin{align*}
				b_2         & = \frac{1}{2} \\
				b_{2 n - 1} & = \frac{(-1)^n}{2 \pi \left( n^2 - n - 1 \right)}
			\end{align*}
			Therefore,
			\begin{align*}
				f(x) & \approx \frac{\sin(2 x)}{2} + \sum\limits_{n = 1}^{\infty} \frac{(-1)^n}{2 \pi \left( n^2 - n - 1 \right)}
			\end{align*}
			$f(x)$ is continuous, $f'(x)$ is piecewise continuous, and $f(-\pi) = f(\pi)$.\\
			Therefore, differentiating term by term,
			\begin{align*}
				f'(x) &= \cos(2 x) + \sum\limits_{n = 1}^{\infty} \frac{(-1)^n n \cos(n x)}{2 \pi \left( n^2 - n - 1 \right)}
			\end{align*}
			Therefore,
			\begin{align*}
				f'\left( -\frac{\pi}{2} \right) & = -1 \\
				f'\left( \frac{\pi}{2} \right)  & = -1
			\end{align*}
	\end{enumerate}
\end{solution}

\begin{question}
	Using the Fourier series of $f(x) = x^2$, calculate the Fourier series of
	\begin{align*}
		g(x) & = x^3 - \pi^2 x
	\end{align*}
	Hint: Use integration by terms.
\end{question}

\begin{solution}
	\begin{align*}
		f(x) & = x^2
	\end{align*}
	$f(x)$ is continuous, $f'(x)$ is piecewise continuous, and $f(-\pi) = f(\pi)$.\\
	Therefore, integrating term by term,
	\begin{align*}
		x^2 & \approx \pi^2 x + 12 \sum\limits_{n = 1}^{\infty} \frac{(-1)^n}{n^3} \sin(n x)
	\end{align*}
	\begin{align*}
		x & \approx 2 \sum\limits_{n = 1}^{\infty} \frac{(-1)^{n + 1} \sin(n x)}{n}
	\end{align*}
	Therefore,
	\begin{align*}
		g(x) & = x^3 - \pi^2 x \\
                     & = 12 \sum\limits_{n = 1}^{\infty} \frac{(-1)^n}{n^3} \sin(n x)
	\end{align*}
\end{solution}

\begin{question}
	Let $f$ be a piecewise continuous function with period $2 \pi$.
	The Fourier series of $f$ is
	\begin{align*}
		f & \approx \sum\limits_{-\infty , n \neq 0}^{\infty} \hat{f}(n) e^{i n x}
	\end{align*}
	Let
	\begin{align*}
		g(x) & = \int\limits_{-\pi}^{\pi} \left( f(t) + f(\pi - t) \right) \dif t
	\end{align*}
	Find the Fourier coefficients $\hat{g}(n)$ using $\hat{f}(n)$.
\end{question}

\begin{solution}
	\begin{align*}
		f(t) & \approx \sum\limits_{n = -\infty}^{\infty} \hat{f}(n) e^{i n x}
	\end{align*}
	Therefore,
	\begin{align*}
		f(\pi - t) & \approx \sum\limits_{n = -\infty}^{\infty} \hat{f}(n) (-1)^n e^{-i n x} \\
                           & \approx \sum\limits_{n = -\infty}^{\infty} \hat{f}(-n) (-1)^{-n} e^{i n x}
	\end{align*}
	Therefore,
	\begin{align*}
		g(x) & = \int\limits_{-\pi}^{x} \left( f(t) + f(\pi - t) \right) \dif t                                                                                                               \\
                     & \approx \int\limits_{-\pi}^{x} \left( \sum\limits_{n = -\infty}^{\infty} \hat{f}(n) e^{i n t} + \sum\limits_{n = -\infty}^{\infty} \hat{f}(-n) (-1)^n e^{i n t} \right) \dif t \\
                     & \approx \sum\limits_{n = \infty}^{\infty} \int\limits_{-\pi}^{x} e^{i n t} \left( \hat{f}(n) + (-1)^n \hat{f}(-n) \right) \dif t                                               \\
                     & = \sum\limits_{n = -\infty}^{\infty} \left. \left( \frac{\hat{f}(n) + (-1)^n \hat{f}(-n)}{i n} \right) \right|_{-\pi}^{x}
	\end{align*}
	Therefore,
	\begin{align*}
		\hat{g}(n) & = \frac{\hat{f}(n) + (-1)^n \hat{f}(-n)}{i n}
	\end{align*}
\end{solution}

\begin{question}
	Let $f$ be the $2 \pi$ periodic function such that
	\begin{align*}
		f(x) & = e^x
	\end{align*}
	for $x \in [-\pi,\pi]$, and $\sum\limits_{n = -\infty}^{\infty} \hat{f}(n) e^{i n x}$ its Fourier series.
	So for $|x| < \pi$, we have
	\begin{align*}
		e^x & = \sum\limits_{-\infty}^{\infty} \hat{f}(n) e^{i n x}
	\end{align*}
	We will formally differentiate both side and get
	\begin{align*}
		e^x & = \sum\limits_{-\infty}^{\infty} i n \hat{f}(n) e^{i n x}
	\end{align*}
	So we have 
	\begin{align*}
		\hat{f}(n)            & = i n \hat{f}(n) \\
		\therefore \hat{f}(n) & = 0
	\end{align*}
	Where was our mistake?
\end{question}

\begin{solution}
	As $f(-\pi) \neq f(\pi)$, term by term differentiation is not possible.
\end{solution}

\end{document}
