\documentclass[fleqn, a4paper, 11pt, oneside]{amsart}
%\usepackage[top = 2cm, bottom = 1cm, left = 1cm, right = 1cm]{geometry}
\usepackage{exsheets, tasks}
\usepackage{amsmath, amssymb, amsthm} %standard AMS packages
\usepackage{marginnote} %marginnotes
\usepackage{gensymb} %miscellaneous symbols
\usepackage{commath} %differential symbols
\usepackage{xcolor} %colours
\usepackage{cancel} %cancelling terms
\usepackage[free-standing-units, space-before-unit]{siunitx} %formatting units
\usepackage{tikz, pgfplots} %diagrams
\usetikzlibrary{calc, hobby, patterns, intersections, decorations.markings}
\usepackage{graphicx} %inserting graphics
\usepackage{hyperref} %hyperlinks
\usepackage{datetime} %date and time
\usepackage{ulem} %underline for \emph{}
\usepackage{xfrac} %inline fractions
\usepackage{enumerate,enumitem} %numbered lists
\usepackage{float} %inserting floats
\usepackage{circuitikz}[american voltages, american currents] %circuit diagrams

\newcommand\numberthis{\addtocounter{equation}{1}\tag{\theequation}} %adds numbers to specific equations in non-numbered list of equations

\newcommand{\AxisRotator}[1][rotate=0]{
	\tikz [x=0.25cm,y=0.60cm,line width=.2ex,-stealth,#1] \draw (0,0) arc (-150:150:1 and 1);%
} %rotation symbols on axes

\theoremstyle{definition}
\newtheorem{example}{Example}
\newtheorem{definition}{Definition}

\theoremstyle{theorem}
\newtheorem{theorem}{Theorem}

\newcommand{\curl}{\mathrm{curl\,}}

\makeatletter
\@addtoreset{section}{part} %resets section numbers in new part
\makeatother

\renewcommand{\thesubsection}{(\arabic{subsection})}
\renewcommand{\thesection}{(\arabic{section})}

%section headings on left
\makeatletter
\def\specialsection{\@startsection{section}{1}%
	\z@{\linespacing\@plus\linespacing}{.5\linespacing}%
	%  {\normalfont\centering}}% DELETED
	{\normalfont}}% NEW
\def\section{\@startsection{section}{1}%
	\z@{.7\linespacing\@plus\linespacing}{.5\linespacing}%
	%  {\normalfont\scshape\centering}}% DELETED
	{\normalfont\scshape}}% NEW
\makeatother

%forces newline after subsection
\makeatletter
\def\subsection{\@startsection{subsection}{3}%
	\z@{.5\linespacing\@plus.7\linespacing}{.1\linespacing}%
	{\normalfont\itshape}}
\makeatother

\settasks{counter-format = tsk[1].}

\SetupExSheets{solution/print = true}

%opening
\title{Harmonic Analysis : Assignment 2}
\author
{
	Aakash Jog\\
	ID : 989323563
}
\date{\formatdate{10}{11}{2015}}

\begin{document}

\tikzset{->-/.style={decoration={
  markings,
  mark=at position #1 with {\arrow{>}}},postaction={decorate}}}

\maketitle
%\setlength{\mathindent}{0pt}

\begin{question}
	Is there a piecewise continuous function $f$ on $[-\pi,\pi]$ such that $b_n = \frac{1}{n}$, $a_n = (-1)^n$?
\end{question}

\begin{solution}
	By Riemann-Lebesgue's Lemma, if $f$ is piecewise continuous, then
	\begin{equation*}
		\lim\limits_{n \to \infty} a_n = \lim\limits_{n \to \infty} = 0
	\end{equation*}
	Therefore, as $\lim\limits_{n \to \infty} a_n$ does not exist, $f$ cannot be piecewise continuous.
\end{solution}

\begin{question}
	Calculate the limit $L = \lim\limits_{n \to \infty} \int\limits_{-\pi}^{\pi} \left( \sin^{\frac{1}{3}} x + \sin n x \right) \sin n x \dif x$.
\end{question}

\begin{solution}
	As $\sin^{\frac{1}{3}} x$ is piecewise continuous, by Riemann-Lebesgue's Lemma,
	\begin{align*}
		L & = \lim\limits_{n \to \infty} \int\limits_{-\pi}^{\pi} \left( \sin^{\frac{1}{3}} x + \sin n x \right) \sin n x \dif x                                                                 \\
                  & = \lim\limits_{n \to \infty} \left( \int\limits_{-\pi}^{\pi} \sin^{\frac{1}{3}} x \sin n x \dif x + \int\limits_{-\pi}^{\pi} \sin^2 n x \dif x \right)                               \\
                  & = \pi \lim\limits_{n \to \infty} \frac{1}{\pi} \int\limits_{-\pi}^{\pi} \sin^{\frac{1}{3}} x \sin n x \dif x + \lim\limits_{n \to \infty} \int\limits_{-\pi}^{\pi} \sin^2 n x \dif x \\
                  & = 0 + \lim\limits_{n \to \infty} \int\limits_{-\pi}^{\pi} \sin^2 n x                                                                                                                 \\
                  & = \frac{1}{2} \lim\limits_{n \to \infty} \int\limits_{-\pi}^{\pi} 1 - \cos 2 n x \dif x                                                                                              \\
                  & = \pi
	\end{align*}
\end{solution}

\begin{question}
	Let $f$ be a piecewise periodic function with period $\beta$.
	Show that the integral $\int\limits_{a}^{a + \beta} f(x) \dif x$ is independent of $a$.
	Hint: Define $g(a) = \int\limits_{a}^{a + \beta} f(x) \dif x$ and show that $g$ is constant.
\end{question}

\begin{solution}
	Let
	\begin{align*}
		g(a) & = \int\limits_{a}^{a + \beta} f(x) \dif x                                   \\
                     & = \int\limits_{a}^{0} f(x) \dif x + \int\limits_{0}^{a + \beta} f(x) \dif x \\
                     & = \int\limits_{0}^{a + \beta} f(x) \dif x - \int\limits_{0}^{a} f(x) \dif x
	\end{align*}
	Therefore,
	\begin{align*}
		g'(a) & = f(a + \beta) - f(a)
	\end{align*}
	As $f$ is periodic, $\forall a$,
	\begin{align*}
		f(a + \beta)     & = f(a) \\
		\therefore g'(a) & = 0
	\end{align*}
	Therefore, $g(a)$ is independent of $a$.
	\qed
\end{solution}

\begin{question}
	Prove that for all $x \in (-\pi,\pi)$, the following equality holds.
	\begin{align*}
		\sum\limits_{n = 1}^{\infty} \frac{(-1)^{n + 1}}{n} \sin n x & = \frac{x}{2}
	\end{align*}
	Hint: Develop the Fourier series of $f(x) = x$ and use Dirichlet theorem to show point wise continuity.
\end{question}

\begin{solution}
	Let
	\begin{align*}
		f(x) & = x
	\end{align*}
	Therefore,
	\begin{align*}
		a_0 & = \frac{1}{\pi} \int\limits_{-\pi}^{\pi} f(x) \dif x             \\
                    & = \frac{1}{\pi} \int\limits_{-\pi}^{\pi} x \dif x                \\
                    & = \frac{1}{\pi} \left. \frac{x^2}{2} \right|_{-\pi}^{\pi}        \\
                    & = \frac{1}{\pi} \left( \frac{\pi^2}{2} - \frac{\pi^2}{2} \right) \\
                    & = 0
	\end{align*}
	\begin{align*}
		a_n & = \frac{1}{\pi} \int\limits_{-\pi}^{\pi} f(x) \cos(n x) \dif x                            \\
                    & = \frac{1}{\pi} \int\limits_{-\pi}^{\pi} x \cos(n x) \dif x                               \\
                    & = \frac{1}{\pi} \left. \frac{\cos(n x)}{n^2} + \frac{x \sin(n x)}{n} \right|_{-\pi}^{\pi} \\
                    & = 0
	\end{align*}
	\begin{align*}
		b_n & = \frac{1}{\pi} \int\limits_{-\pi}^{\pi} f(x) \sin(n x) \dif x                            \\
                    & = \frac{1}{\pi} \int\limits_{-\pi}^{\pi} x \sin(n x) \dif x                               \\
                    & = \frac{1}{\pi} \left. \frac{\sin(n x)}{n^2} - \frac{x \cos(n x)}{n} \right|_{-\pi}^{\pi} \\
                    & = 2 \frac{(-1)^{n + 1}}{n}
	\end{align*}
	Therefore,
	\begin{align*}
		x & \approx 2 \sum\limits_{n = 1}^{\infty} (-1)^{n + 1} \frac{\sin(n x)}{n}
	\end{align*}
	Therefore, by Dirichlet theorem,
	\begin{align*}
		2 \sum\limits_{n = 1}^{\infty} \frac{(-1)^{n + 1}}{n} \sin n x          & = \frac{f(x^-) + f(x^+)}{2} \\
                                                                                        & = x                         \\
		\therefore \sum\limits_{n = 1}^{\infty} \frac{(-1)^{n + 1}}{n} \sin n x & = \frac{x}{2}
	\end{align*}
	\qed
\end{solution}

\begin{question}
	Using the function $g(x) = x \left( \pi - |x| \right)$ calculate the series $\sum\limits_{n = 1}^{\infty} \frac{(-1)^{n + 1}}{(2 n - 1)^3}$.
\end{question}

\begin{solution}
	\begin{align*}
		g(x) & = x \left( \pi - |x| \right)
	\end{align*}
	Therefore, as the function is odd, its Fourier series is
	\begin{align*}
		g(x) & \approx \sum\limits_{n = 1}^{\infty} b_n \sin(n x) \dif x
	\end{align*}
	Therefore,
	\begin{align*}
		b_n & = \frac{1}{\pi} \int\limits_{-\pi}^{\pi} x \left( \pi - |x| \right) \sin(n x) \dif x                                                   \\
                    & = \frac{1}{\pi} \int\limits_{-\pi}^{0} x (\pi + x) \sin(n x) \dif x + \frac{1}{\pi} \int\limits_{0}^{\pi} x (\pi - x) \sin(n x) \dif x \\
                    & = \frac{1}{\pi} \frac{2 - 2 \cos(n \pi) - n \pi \sin(n \pi)}{n^3} + \frac{1}{\pi} \frac{2 - 2 \cos(n \pi) - n \pi \sin(n \pi)}{n^3}    \\
                    & = \frac{4}{\pi n^3} \left( 1 - (-1)^n \right)
	\end{align*}
	Therefore,
	\begin{align*}
		b_n &=
			\begin{cases}
				0                 & ;\quad n \text{ is even} \\
				\frac{8}{\pi n^3} & ;\quad n \text{ is odd}  \\
			\end{cases}
	\end{align*}
	Therefore,
	\begin{align*}
		\frac{g\left( {\frac{\pi}{2}}^+ \right) + g\left( {\frac{\pi}{2}}^- \right)}{2} & = \frac{8}{\pi} \sum\limits_{k = 1}^{\infty} \frac{1}{(2 k - 1)^2} \sin\left( (2 k - 1) x \right) \\
	\end{align*}
	Therefore,
	\begin{align*}
		\sum\limits_{k = 1}^{\infty} \frac{(-1)^{n + 1}}{(2 k - 1)^3} & = \frac{\pi^3}{32}
	\end{align*}
\end{solution}

\begin{question}
	Find the Fourier series of $f(x) = 1 - x^2$.
	For which values does it converge when $x = 5 \pi$ and $x = 6 \pi$.
\end{question}

\begin{solution}
	As $x^2$ is even, its Fourier series is
	\begin{align*}
		x^2 & \approx \frac{a_0}{2} \sum\limits_{n = 1}^{\infty} a_n \cos(n x)
	\end{align*}
	Therefore,
	\begin{align*}
		a_0 & = \frac{1}{\pi} \int\limits_{-\pi}^{\pi} x^2 \dif x       \\
                    & = \frac{1}{\pi} \left. \frac{x^3}{3} \right|_{-\pi}^{\pi} \\
                    & = \frac{1}{\pi} \frac{2 \pi^3}{3}                         \\
                    & = \frac{2 \pi^2}{3}
	\end{align*}
	\begin{align*}
		a_n & = \frac{1}{\pi} \int\limits_{-\pi}^{\pi} x^2 \cos(n x) \dif x                                                            \\
                    & = \frac{1}{\pi} \left. \frac{2 x \cos(n x)}{n^2} + \frac{\left( n^2 x^2 - 2 \right) \sin(n x)}{n^3} \right|_{-\pi}^{\pi} \\
                    & = \frac{4}{n^2} (-1)^n
	\end{align*}
	Therefore,
	\begin{align*}
		x^2                & \approx \frac{\pi^2}{3} + \sum\limits_{n = 1}^{\infty} (-1)^n \frac{4}{n^2} \cos(n x)     \\
		\therefore 1 - x^2 & \approx 1 - \frac{\pi^2}{3} - \sum\limits_{n = 1}^{\infty} (-1)^n \frac{4}{n^2} \cos(n x) \\
	\end{align*}
\end{solution}

\begin{question}
	Out of the Fourier series of $\cos \alpha x$ for non-integral $\alpha$, find $\cot(\alpha \pi)$.
\end{question}

\begin{solution}
	As $\cos(\alpha x)$ is even, its Fourier series is
	\begin{align*}
		\cos(\alpha x) & \approx \frac{a_0}{2} \sum\limits_{n = 1}^{\infty} a_n \cos(n x)
	\end{align*}
	\begin{align*}
		a_0 & = \frac{1}{\pi} \int\limits_{-\pi}^{\pi} \cos(\alpha x) \dif x            \\
                    & = \frac{1}{\pi} \left. \frac{\sin(\alpha x)}{\alpha} \right|_{-\pi}^{\pi} \\
                    & = \frac{2 \sin(\alpha \pi)}{\alpha \pi}
	\end{align*}
	\begin{align*}
		a_n & = \frac{1}{\pi} \int\limits_{-\pi}^{\pi} \cos(\alpha x) \cos(n x) \dif x \\
                    & = \frac{(-1)^n}{\pi} \sin(\alpha \pi) \frac{2 \alpha}{\alpha^2 - n^2}
	\end{align*}
	Therefore,
	\begin{align*}
		\cos(\alpha x) & \approx \frac{\sin(\alpha \pi)}{\alpha \pi} + \sum\limits_{n = 1}^{\infty} \frac{(-1)^n}{\pi} \sin(\alpha \pi) \frac{2 \alpha}{\alpha^2 - n^2} \cos(n x) \\
                               & = \frac{\sin(\alpha \pi)}{\alpha \pi} + \frac{2 \alpha \sin(\alpha \pi)}{\pi} \sum\limits_{n = 1}^{\infty} \frac{(-1)^n}{\alpha^2 - n^2} \cos(n x)
	\end{align*}
	Therefore,
	\begin{align*}
		\cos(\alpha \pi)            & = \frac{\sin(\alpha \pi)}{\alpha \pi} + \frac{2 \alpha \sin(\alpha \pi)}{\pi} \sum\limits_{n = 1}^{\infty} \frac{(-1)^n}{\alpha^2 - n^2} (-1)^n \\
                                            & = \frac{\sin(\alpha \pi)}{\alpha \pi} + \frac{2 \alpha \sin(\alpha \pi)}{\pi} \sum\limits_{n = 1}^{\infty} \frac{1}{\alpha^2 - n^2}             \\
		\therefore \cot(\alpha \pi) & = \frac{1}{\alpha \pi} + \frac{2 \alpha}{\pi} \sum\limits_{n = 1}^{\infty} \frac{1}{\alpha^2 - n^2}
	\end{align*}
\end{solution}

\end{document}
