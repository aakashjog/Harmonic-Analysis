\documentclass[fleqn, a4paper, 11pt, oneside]{amsart}
%\usepackage[top = 2cm, bottom = 1cm, left = 1cm, right = 1cm]{geometry}
\usepackage{exsheets, tasks}
\usepackage{amsmath, amssymb, amsthm} %standard AMS packages
\usepackage{marginnote} %marginnotes
\usepackage{gensymb} %miscellaneous symbols
\usepackage{commath} %differential symbols
\usepackage{xcolor} %colours
\usepackage{cancel} %cancelling terms
\usepackage[free-standing-units, space-before-unit]{siunitx} %formatting units
\usepackage{tikz, pgfplots} %diagrams
\usetikzlibrary{calc, hobby, patterns, intersections, decorations.markings}
\usepackage{graphicx} %inserting graphics
\usepackage{hyperref} %hyperlinks
\usepackage{datetime} %date and time
\usepackage{ulem} %underline for \emph{}
\usepackage{xfrac} %inline fractions
\usepackage{enumerate,enumitem} %numbered lists
\usepackage{float} %inserting floats
\usepackage{circuitikz}[american voltages, american currents] %circuit diagrams

\newcommand\numberthis{\addtocounter{equation}{1}\tag{\theequation}} %adds numbers to specific equations in non-numbered list of equations

\newcommand{\AxisRotator}[1][rotate=0]{
	\tikz [x=0.25cm,y=0.60cm,line width=.2ex,-stealth,#1] \draw (0,0) arc (-150:150:1 and 1);%
} %rotation symbols on axes

\theoremstyle{definition}
\newtheorem{example}{Example}
\newtheorem{definition}{Definition}

\theoremstyle{theorem}
\newtheorem{theorem}{Theorem}

\newcommand{\curl}{\mathrm{curl\,}}

\makeatletter
\@addtoreset{section}{part} %resets section numbers in new part
\makeatother

\renewcommand{\thesubsection}{(\arabic{subsection})}
\renewcommand{\thesection}{(\arabic{section})}

%section headings on left
\makeatletter
\def\specialsection{\@startsection{section}{1}%
	\z@{\linespacing\@plus\linespacing}{.5\linespacing}%
	%  {\normalfont\centering}}% DELETED
	{\normalfont}}% NEW
\def\section{\@startsection{section}{1}%
	\z@{.7\linespacing\@plus\linespacing}{.5\linespacing}%
	%  {\normalfont\scshape\centering}}% DELETED
	{\normalfont\scshape}}% NEW
\makeatother

%forces newline after subsection
\makeatletter
\def\subsection{\@startsection{subsection}{3}%
	\z@{.5\linespacing\@plus.7\linespacing}{.1\linespacing}%
	{\normalfont\itshape}}
\makeatother

\settasks{counter-format = tsk[1].}

\SetupExSheets{solution/print = true}

%opening
\title{Harmonic Analysis : Assignment 7}
\author
{
	Aakash Jog\\
	ID : 989323563
}
\date{\formatdate{22}{12}{2015}}

\begin{document}

\tikzset{->-/.style={decoration={
  markings,
  mark=at position #1 with {\arrow{>}}},postaction={decorate}}}

\maketitle
%\setlength{\mathindent}{0pt}

\begin{question}
	Consider $C[-1,2]$, the space of all complex continuous functions on $[-1,2]$.
	Which of the following expressions define inner product on $C[-1,2]$?
	Explain.
	\begin{enumerate}
		\item $\langle f,g \rangle = \int\limits_{-1}^{2} \left| f(t) + g(t) \right| \dif t$
		\item $\langle f,g \rangle = \int\limits_{-1}^{2} f(t) \overline{g(t)} \dif t + f\left( -\frac{1}{2} \right) \overline{g\left( -\frac{1}{2} \right)}$
		\item $\langle f,g \rangle = f(0) \overline{g(0)} + f(1) \overline{g(1)}$
	\end{enumerate}
\end{question}

\begin{solution}
	\begin{enumerate}
		\item
			\begin{align*}
				\langle f,g \rangle & = \int\limits_{-1}^{2} \left| f(t) + g(t) \right| \dif t
			\end{align*}
			Therefore,
			\begin{align*}
				\overline{\langle g,f \rangle} & = \int\limits_{-1}^{2} \overline{\left| f(t) + g(t) \right|} \dif t \\
                                                               & = \int\limits_{-1}^{2} \left| f(t) + g(t) \right|                   \\
                                                               & = \langle f,g \rangle
			\end{align*}
			Therefore,
			\begin{align*}
				\langle f + h , g \rangle            & = \int\limits_{-1}^{2} \left| f(t) + h(t) + g(t) \right| \dif t                          \\
                                                                     & \neq \int\limits_{-1}^{2} \left| f(t) + g(t) \right| + \left| h(t) + g(t) \right| \dif t \\
				\therefore \langle f + h , g \rangle & \neq \langle f,g \rangle + \langle h,
			\end{align*}
			Therefore, it is not an inner product.
		\item
			\begin{align*}
				\langle f,g \rangle & = \int\limits_{-1}^{2} f(t) \overline{g(t)} \dif t + f\left( -\frac{1}{2} \right) \overline{g\left( -\frac{1}{2} \right)}
			\end{align*}
			Therefore,
			\begin{align*}
				\overline{\langle g,f \rangle} & = \int\limits_{-1}^{2} \overline{g(t) \overline{f(t)}} \dif t + \overline{g\left( -\frac{1}{2} \right) \overline{f\left( -\frac{1}{2} \right)}} \\
                                                               & = \int\limits_{-1}^{2} f(t) \overline{g(t)} \dif t + f\left( -\frac{1}{2} \right) \overline{g\left( -\frac{1}{2} \right)}                       \\
                                                               & = \langle f,g \rangle
			\end{align*}
			Therefore,
			\begin{align*}
				\langle f + h , g \rangle & = \int\limits_{-1}^{2} \left( f(t) + h(t) \right) \overline{g(t)} \dif t + \left( f\left( -\frac{1}{2} \right) + h\left( -\frac{1}{2} \right) \right) \overline{g\left( -\frac{1}{2} \right)}                                                       \\
                                                          & = \int\limits_{-1}^{2} f(t) \overline{g(t)} \dif t + f\left( -\frac{1}{2} \right) \overline{g\left( -\frac{1}{2} \right)} + \int\limits_{-1}^{2} h(t) \overline{g(t)} \dif t + h\left( -\frac{1}{2} \right) \overline{g\left( -\frac{1}{2} \right)} \\
                                                          & = \langle f,g \rangle + \langle h,g \rangle
			\end{align*}
			Therefore,
			\begin{align*}
				\langle \alpha f , g \rangle & = \int\limits_{-1}^{2} \alpha f(t) \overline{g(t)} \dif t + \alpha f\left( -\frac{1}{2} \right) \overline{g\left( -\frac{1}{2} \right)} \\
                                                             & = \alpha \langle f,g \rangle
			\end{align*}
			Therefore,
			\begin{align*}
				\langle f,f \rangle & = \int\limits_{-1}^{2} f(t) \overline{f(t)} \dif t + f\left( -\frac{1}{2} \right) \overline{f\left( -\frac{1}{2} \right)} \\
                                                    & = \int\limits_{-1}^{2} \left| f(t) \right|^2 \dif t + \left| f\left( -\frac{1}{2} \right) \right|^2                       \\
                                                    & \ge 0
			\end{align*}
			Therefore, it is an inner product.
		\item
			\begin{align*}
				\langle f,g \rangle & = f(0) \overline{g(0)} + f(1) \overline{g(1)}
			\end{align*}
			Therefore,
			\begin{align*}
				\overline{\langle g,f \rangle} & = \overline{g(0) \overline{f(0)} + g(1) \overline{f(1)}}            \\
                                                               & = \overline{g(0) \overline{f(0)}} + \overline{g(1) \overline{f(1)}} \\
                                                               & = f(0) \overline{g(0)} + f(1) \overline{g(1)}                       \\
                                                               & = \langle f,g \rangle
			\end{align*}
			Therefore,
			\begin{align*}
				\langle f + h , g \rangle & = \left( f(0) + h(0) \right) \overline{g(0)} + \left( f(1) + h(1) \right) \overline{g(1)}   \\
                                                          & = f(0) \overline{g(0)} + f(1) \overline{g(1)} + h(0) \overline{g(0)} + h(1) \overline{g(1)} \\
                                                          & = \langle f,g \rangle + \langle h,g \rangle
			\end{align*}
			Therefore,
			\begin{align*}
				\langle \alpha f , g \rangle & = \alpha f(0) \overline{g(0)} + \alpha f(1) \overline{g(1)} \\
                                                             & = \alpha \langle f,g \rangle
			\end{align*}
			Therefore,
			\begin{align*}
				\langle f,f \rangle & = f(0) \overline{f(0)} + f(1) \overline{f(1)}   \\
                                                    & = \left| f(0) \right|^2 + \left| f(1) \right|^2 \\
                                                    & \ge 0
			\end{align*}
			Therefore, it is an inner product.
	\end{enumerate}
\end{solution}

\begin{question}
	Let $V$ be the space of all real, twice continuously differentiable functions of $[-\pi,\pi]$.
	Is
	\begin{align*}
		\langle f,g \rangle & = f(-\pi) g(-\pi) + \int\limits_{-\pi}^{\pi} f''(x) g''(x) \dif x
	\end{align*}
	an inner product on $V$?
\end{question}

\begin{solution}
	\begin{align*}
		\langle f,g \rangle & = f(-\pi) g(-\pi) + \int\limits_{-\pi}^{\pi} f''(x) g''(x) \dif x
	\end{align*}
	Therefore,
	\begin{align*}
		\overline{\langle g,f \rangle} & = \overline{g(-\pi) f(-\pi)} + \int\limits_{-\pi}^{\pi} \overline{f''(x) g''(x)} \dif x \\
                                               & = g(-\pi) f(-\pi) + \int\limits_{-\pi}^{\pi} f''(x) g''(x) \dif x                       \\
                                               & = \langle f,g \rangle
	\end{align*}
	Therefore,
	\begin{align*}
		\langle f + h , g \rangle\ & = \left( f(-\pi) + h(-\pi) \right) g(-\pi) + \int\limits_{-\pi}^{\pi} \left( f''(x) + h''(x) \right) g''(x) \dif x                                                    \\
                                           & = f(-\pi) g(-\pi) + \int\limits_{-\pi}^{\pi} \left( f''(x) + g''(x) \right) \dif x + h(-\pi) g(-\pi) + \int\limits_{-\pi}^{\pi} \left( h''(x) + g''(x) \right) \dif x \\
                                           & = \langle f,g \rangle + \langle h,g \rangle
	\end{align*}
	Therefore,
	\begin{align*}
		\langle \alpha f , g \rangle & = \alpha f(-\pi) g(-\pi) + \int\limits_{-\pi}^{\pi} \left( \alpha f''(x) + g''(x) \right) \dif x \\
                                             & = \alpha \langle f,g \rangle
	\end{align*}
	Therefore,
	\begin{align*}
		\langle f,f \rangle & = f(-\pi) f(-\pi) + \int\limits_{-\pi}^{\pi} f''(x) f''(x) \dif x                    \\
                                    & = \left( f(-\pi) \right)^2 + \int\limits_{-\pi}^{\pi} \left( f''(x) \right)^2 \dif x \\
                                    & \ge 0
	\end{align*}
	Therefore, it is an inner product.
\end{solution}

\begin{question}
	Consider $C^1[0,1]$, the space of all complex continuously differentiable functions on $[0,1]$.
	Which of the following expressions define inner product on $C^1[0,1]$?
	Explain.
	\begin{enumerate}
		\item $\langle f,g \rangle = f(0) \overline{g(0)} + \int\limits_{0}^{1} f'(t) \overline{g'(t)} \dif t$
		\item $\langle f,g \rangle = f(0) \overline{g(0)} + f'(1) \overline{g'(1)}$
	\end{enumerate}
\end{question}

\begin{solution}
	\begin{enumerate}[leftmargin=*]
		\item
			\begin{align*}
				\langle f,g \rangle & = f(0) \overline{g(0)} + \int\limits_{0}^{1} f'(t) \overline{g'(t)} \dif t
			\end{align*}
			Therefore,
			\begin{align*}
				\overline{\langle g,f \rangle} & = \overline{g(0) f(0)} + \int\limits_{0}^{1} \overline{g'(t) \overline{f'(t)}} \dif t \\
                                                               & = f(0) \overline{g(0)} + \int\limits_{0}^{1} f'(t) \overline{g'(t)} \dif t            \\
                                                               & = \langle f,g \rangle
			\end{align*}
			Therefore,
			\begin{align*}
				\langle f + g , h \rangle & = \left( f(0) + h(0) \right) \overline{g(0)} + \int\limits_{0}^{1} \left( f'(t) + h'(t) \right) \overline{g'(t)} \dif t                                 \\
                                                          & = f(0) \overline{g(0)} + h(0) + \overline{g(0)} + \int\limits_{0}^{1} f'(t) \overline{g'(t)} \dif t + \int\limits_{0}^{1} h'(t) \overline{g'(t)} \dif t \\
                                                          & = \langle f,g \rangle + \langle h,g \rangle
			\end{align*}
			Therefore,
			\begin{align*}
				\langle \alpha f , g \rangle & = \alpha f(0) \overline{g(0)} + \int\limits_{0}^{1} \alpha f'(t) \overline{g'(t)} \dif t \\
                                                             & = \alpha \langle f,g \rangle
			\end{align*}
			Therefore,
			\begin{align*}
				\langle f,f \rangle & = f(0) \overline{f(0)} + \int\limits_{0}^{1} f'(t) \overline{f'(t)} \dif t  \\
                                                    & = \left| f(0) \right|^2 + \int\limits_{0}^{1} \left| f'(t) \right|^2 \dif t \\
                                                    & \ge 0
			\end{align*}
			Therefore, it is an inner product.
		\item
			\begin{align*}
				\langle f,g \rangle & = f(0) \overline{g(0)} + f'(1) \overline{g'(1)}
			\end{align*}
			Therefore,
			\begin{align*}
				\overline{\langle g,f \rangle} & = \overline{g(0) \overline{f(0)} + g'(1) \overline{f'(1)}}            \\
                                                               & = \overline{g(0) \overline{f(0)}} + \overline{g'(1) \overline{f'(1)}} \\
                                                               & = f(0) \overline{g(0)} + f'(1) \overline{g'(1)}                       \\
                                                               & = \langle f,g \rangle
			\end{align*}
			Therefore,
			\begin{align*}
				\langle f + h , g \rangle & = \left( f(0) + h(0) \right) \overline{g(0)} + \left( f'(1) + h'(1) \right) \overline{g'(1)}    \\
                                                          & = f(0) \overline{g(0)} + h(0) \overline{g(0)} + f'(1) \overline{g'(1)} + h'(1) \overline{g'(1)} \\
                                                          & = \langle f,g \rangle + \langle h,g \rangle
			\end{align*}
			Therefore,
			\begin{align*}
				\langle \alpha f , g \rangle & = \alpha f(0) \overline{g(0)} + \alpha f'(1) \overline{g'(1)} \\
                                                             & = \alpha \langle f,g \rangle
			\end{align*}
			Therefore,
			\begin{align*}
				\langle f,f \rangle & = f(0) \overline{f(0)} + f'(1) \overline{f'(1)}  \\
                                                    & = \left| f(0) \right|^2 + \left| f'(1)^2 \right| \\
                                                    & \ge 0
			\end{align*}
			Therefore, it is an inner product.
	\end{enumerate}
\end{solution}

\begin{question}
	Let $V$ be an inner product space.
	Prove that for all $u,v \in V$,
	\begin{align*}
		\langle u,v \rangle & = \frac{1}{4} \|u + v\|^2 - \frac{1}{4} \|u - v\|^2
	\end{align*}
\end{question}

\begin{solution}
	\begin{align*}
		\frac{1}{4} \|u + v\|^2 - \frac{1}{4} \|u - v\|^2 & = \frac{1}{4} \left( \|u + v\|^2 - \|u - v\|^2 \right)                                                                                                                                                                                    \\
                                                                  & = \frac{1}{4} \left( \langle u + v , u + v \rangle - \langle u - v , u - v \rangle \right)                                                                                                                                                \\
                                                                  & = \frac{1}{4} \left( \left( \langle u,u \rangle  + \langle u,v \rangle + \langle v,u \rangle + \langle v,v \rangle \right) - \left( \langle u,u \rangle - \langle u,v \rangle - \langle v,u \rangle + \langle v,v \rangle \right) \right) \\
                                                                  & = \frac{1}{4} \left( 2 \left( \langle u,v \rangle + \langle v,u \rangle \right) \right)                                                                                                                                                   \\
                                                                  & = \frac{1}{2} \left( \langle u,v \rangle + \overline{\langle u,v \rangle} \right)                                                                                                                                                         \\
                                                                  & = \Re\left( \langle u,v \rangle \right)
	\end{align*}
	Therefore, if $V$ is a subset of $\mathbb{R}$, then the equality holds.
\end{solution}

\end{document}
